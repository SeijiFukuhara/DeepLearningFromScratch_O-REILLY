\section{Python入門}
\subsection{Pythonとは}

\subsection{Pythonのインストール}

\subsection{Pythonインタプリタ}

\subsection{Pythonスクリプトファイル}
\subsubsection{ファイルに保存}

\subsubsection{クラス}
\lref{lst:1_Man}にクラスを実装する例を示します。
ここでは、Manという新しいクラスを定義しています。上の例では、
このManというクラスから、mというインスタンス(オブジェクト)を生成します。
Manクラスのコンストラクタ(初期化メソッド)は、nameという引数を取り、その引数でインスタンス変数であるself.nameを初期化します。\textbf{インスタンス変数}とは、個々のインスタンスに格納される変数のことです。Pythonでは、self.nameのように、selfの後に属性名を書くことでインスタンス変数の作成およびアクセスができます。

\subsection{Numpy}

\subsection{Matplotlib}

\subsection{まとめ}

\subsection{ソースコード}
1章のソースコードを以下に示します。

\ShowPython{Man.py}